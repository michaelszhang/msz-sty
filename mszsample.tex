\documentclass{mszreport}

\title{Sample Document}

\begin{document}

\maketitle

\chapter{Boxes}

\begin{theorem}
    A sample theorem.
\end{theorem}

\begin{lemma}
    A sample lemma.
\end{lemma}

\begin{claim}
    A sample claim.
\end{claim}

\begin{proposition}
    A sample proposition.
\end{proposition}

\begin{corollary}
    A sample corollary.
\end{corollary}

\begin{conjecture}
    A sample conjecture.
\end{conjecture}

\begin{algorithm}
    A sample algorithm.
\end{algorithm}

\begin{definition}
    A sample definition.
\end{definition}

\begin{example}
    A sample example.
\end{example}

\begin{fact}
    A sample fact.
\end{fact}

\begin{note}
    A sample note.
\end{note}

\begin{problem}
    A sample problem.
\end{problem}

\begin{question}
    A sample question.
\end{question}

\begin{exercise}
    A sample exercise.
\end{exercise}

\begin{remark}
    A sample remark.
\end{remark}

\chapter{Commands}

\section{Emphasizing}

\begin{itemize}
    \item \verb|\vocab| may be used to \vocab{bold and change color}.
    \item \verb|\answer| may be used to \answer{change color}.
\end{itemize}

\section{Mathematical fonts}

\begin{itemize}
    \item \verb|\mbf{A}| may be used for \verb|\mathbf{A}|: $\mbf{A}$.
    \item \verb|\mbb{A}| may be used for \verb|\mathbb{A}|: $\mbb{A}$.
    \item \verb|\mcl{A}| may be used for \verb|\mathcal{A}|: $\mcl{A}$.
    \item \verb|\mrm{A}| may be used for \verb|\mathrm{A}|: $\mrm{A}$.
    \item \verb|\tx{A}| may be used for \verb|\text{A}|: $\tx{A}$.
\end{itemize}

\section{Delimiters}

\begin{itemize}
    \item \verb|\braces{}| may be used for $\braces{\tx{braces}}$.
    \item \verb|\parens{}| may be used for $\parens{\tx{parentheses}}$.
    \item \verb|\brackets{}| may be used for $\brackets{\tx{brackets}}$.
    \item \verb|\bbrackets{}| may be used for $\bbrackets{\tx{double brackets}}$.
    \item \verb|\angles{}| may be used for $\angles{\tx{angle brackets}}$.
    \item \verb|\verts{}| may be used for $\verts{\tx{vertical bars}}$.
    \item \verb|\Verts{}| may be used for $\Verts{\tx{double vertical bars}}$.
    \item \verb|\floor{}| may be used for $\floor{\tx{floor delimiters}}$.
    \item \verb|\ceil{}| may be used for $\ceil{\tx{ceiling delimiters}}$.
\end{itemize}

\section{Ordinal numbers}

\begin{itemize}
    \item \verb|\onth| may be used to denote superscript th, as in $0\onth$.
    \item \verb|\onst| may be used to denote superscript st, as in $1\onst$.
    \item \verb|\onnd| may be used to denote superscript nd, as in $2\onnd$.
    \item \verb|\onrd| may be used to denote superscript rd, as in $3\onrd$.
\end{itemize}

\section{General}

The following operators may be used:
\begin{itemize}
    \item \verb|\argmin| for $\argmin$,
    \item \verb|\argmax| for $\argmax$,
    \item \verb|\Re| for $\Re$,
    \item \verb|\Im| for $\Im$,
    \item \verb|\cis| for $\cis$,
    \item \verb|\arcsinh| for $\arcsinh$,
    \item \verb|\arccosh| for $\arccosh$,
    \item \verb|\arctanh| for $\arctanh$, and
    \item \verb|\sign| for $\sign$.
\end{itemize}

\section{Statistics}

The following operators may be used:
\begin{itemize}
    \item \verb|\Prb| for the probability operator $\Prb$,
    \item \verb|\Exp| for the expectation operator $\Exp$,
    \item \verb|\Var| for the variance operator $\Var$, and
    \item \verb|\Cov| for the covariance operator $\Cov$.
\end{itemize}

\section{Calculus}

\begin{itemize}
    \item \verb|\dv{f}{x}| may be used for a first derivative: $\dv{f}{x}$.
    \item \verb|\ddv{f}{x}| may be used for a second derivative: $\ddv{f}{x}$.
    \item \verb|\dnv{f}{x}{n}| may be used for an $n\onth$ derivative: $\dnv{f}{x}{n}$.
    \item \verb|\pdv{f}{x}| may be used for a first partial derivative: $\pdv{f}{x}$.
    \item \verb|\pddv{f}{x}| may be used for a second partial derivative: $\pddv{f}{x}$.
    \item \verb|\pdnv{f}{x}{n}| may be used for an $n\onth$ partial derivative: $\pdnv{f}{x}{n}$.
    \item \verb|\grad| may be used to denote the gradient operator: $\grad{f}$.
    \item \verb|\div| may be used to denote the divergence operator: $\div{f}$.
    \item \verb|\curl| may be used to denote the curl operator: $\curl{f}$.
\end{itemize}

\end{document}